\documentclass[hidelinks]{resume} % Use the custom resume.cls style

% \usepackage[left=0.4 in,top=0.2in,right=0.4 in,bottom=0.2in]{geometry} % Document margins
\usepackage[left=0.5in, top=0.4in,right=0.5in, bottom=0.4in]{geometry} % Document margins

% Using the `raleway` font.
\usepackage[T1]{fontenc}
\usepackage[default]{raleway}

\usepackage{hyperref}

\usepackage{xcolor}

\usepackage{enumitem}
\setlist[itemize]{noitemsep, before=\vspace{-6pt}, leftmargin=4mm}
% \setlist[itemize]{topsep=-1pt}
% \setlist[itemize]{noitemsep}
% \setlist[itemize]{nosep, after=\vspace{\baselineskip}}
% \setlist[itemize]{nosep, before=\vspace{-4pt}, after=\vspace{0pt}}

\newcommand{\tab}[1]{\hspace{.2667\textwidth}\rlap{#1}} 
\newcommand{\itab}[1]{\hspace{0em}\rlap{#1}}
\name{dennis akar} % Your name
%\address{123 Pleasant Lane \\ City, State 12345} % Your secondary addess (optional)
% \address{\href{mailto:da537@cam.ac.uk}{da537@cam.ac.uk} \\
\address{\href{mailto:denizhanak@gmail.com}{denizhanak@gmail.com} \\
% \href{https://denizhanakar.com}{denizhanakar.com} \\
\href{https://denden.dev}{denden.dev} \\
\href{https://github.com/dendenakar}{github.com/dendenakar} \\
UK Citizen}  % Your phone number and email
% denizhanak@gmail.com\\
% github.com/denizhanakar

\begin{document}
% ============================================================================ %
% EDUCATION
\begin{rSection}{Education}

{\textbf{University of Cambridge}\hfill {2021 - 2022}\\
\textit{MPhil in Advanced Computer Science}}
\begin{itemize}
    \item Pass with distinction with 81.20\% GPA.
    \item Awarded the £5,000 ACS MPhil Scholarship for academic excellence.
    \item Researched geometric DL for molecular graphs (drug discovery) supervised by Prof Pietro Liò \& Dr Cristian Bodnar.
\end{itemize}

{\textbf{University of Manchester}} \hfill {2018 - 2021}\\
\textit{BSc Computer Science and Mathematics}
\begin{itemize}
    \item First Class Honours with 83.36\% GPA.
    \item Awarded Certificate of Excellence for top 10\% graduating students.
\end{itemize}

\end{rSection}
% \input{2skills}
% ============================================================================ %
% EXPERIENCE
\begin{rSection}{Experience}

\textbf{MATS: Foundations of Mechinterp (Lee Sharkey) - Researcher}
\hfill {May 2023 - Present}
\begin{itemize}
    \item Investigating ``Attention Head Superposition'' in language models with Chris Mathwin and Lee Sharkey.
    \item Discovered how certain properties of the model affect model performance. 
    \item Facilitated Alignment 201 reading group for 5 MATS scholars.
\end{itemize}

\textbf{ARENA}
\hfill{June 2023}
\begin{itemize}
    \item Self-studied Redwood Research's MLAB curriculum to work as a TA (for the mechanistic interpretability chapter) and as a participant (for the training LLMs at scale chapter) for ARENA (Alignment Research Engineer Accelerator).
    \item Aided participants on examining and interpreteing Indirect Object Identification, balanced bracket classification, superposition, and OthelloGPT.
\end{itemize}

\textbf{MATS: Mechanistic Interpretability (Neel Nanda) - Researcher}
\hfill {Nov 2022 - Jan 2023}
\begin{itemize}
    \item Applied the original and extended logit lens to the IOI task across a set of GPT-2 sized models (extended DLA). Extended logit lens uses consecutive layers at the end of the model to map the residual stream to logit space.
    \item Found the tendency for certain models (e.g. GPT-Neo) to ``flip'' i.e. assign an \textit{extremely low probability} throughout the model to the token that it will eventually output and used extended DLA to analyze how this tendency changes.
\end{itemize}

\textbf{CancerAI (University of Cambridge) - Research Assistant}
\hfill {Jul 2022 - Oct 2022}
% Improved performance of the precision oncology platform OncOS.\\
\begin{itemize}
    \item Researched explainable AI for use by clinical oncologists using \textbf{Tensorflow} and \textbf{PyTorch}.
    \item Developed front-end for VIIDA, an application for analyzing, modelling, explaining, and predicting cancer-related data with \textbf{Flask} and \textbf{React}.
\end{itemize}

\textbf{Cambridge Cancer Genomics - Software Engineer Intern}
\hfill {Jun 2019 - Sep 2019}
% Improved performance of the precision oncology platform OncOS.\\
\begin{itemize}
    \item Integrated features and fixed bugs for the precision oncology platform OncOS backend using \textbf{Python} and \textbf{Flask}.
    \item Built a \textbf{full-stack} internal monitoring system for OncOS infrastructure to manage genomic data and processes.
    \item Researched variational autoencoder algorithms related to DNA sequence compression for SomaticNET, a neural network for evaluating tumor variants, using \textbf{Tensorflow (Python)}, \textbf{Bash}, \textbf{pysam} and \textbf{Annoy}.
\end{itemize}

\end{rSection} 
% ============================================================================ %
% PROJECTS
\begin{rSection}{Projects}

\textbf{Geometric CW Networks} - MPhil Thesis
\hfill{2021 - 2022}
\begin{itemize}
    \item Introduced geometric inductive priors [E(3) invariance and equivariance] to a GNN with a topological inductive prior, in this case CW Networks (CWNs), an architecture in which graphs are ``lifted'' into higher order hypergraphs using CW complexes, to construct Geometric CW Networks (GCWNs).
    \item Used \textbf{PyTorch}, \textbf{PyTorch Geometric}, \textbf{gudhi}.
\end{itemize}

\textbf{Topological Neural Processes} - BSc Thesis
\hfill{2020 - 2021}
\begin{itemize}
    \item Built a novel machine learning model for extracting latent information of topological structures of input (topological data analysis) for Conditional Neural Processes (a neural model which meta-learns a stochastic process) using \textbf{Tensorflow}, \textbf{matplotlib}, \textbf{pickle}, \textbf{gudhi}, and \textbf{numpy}; supervised by Dr Tingting Mu and Dr Cristian Bodnar.
\end{itemize}

\end{rSection}
% ============================================================================ %
% LEADERSHIP
\begin{rSection}{Leadership}

\textbf{AI Safety Facilitator and Tutor}
\hfill{2023}
\begin{itemize}
    \item Facilitating AI Safety Fundamentals 101 and 201 reading groups and tutoring for MLAB/ARENA (including my modifications for Anki support) for MATS research fellows and graduates from the University of Cambridge.
\end{itemize}

\end{rSection}
\end{document}

% ============================================================================ %